\section{Problem Formulation} \label{sec:probform}  

We study systems of quadratic equations of the form
\begin{align}
& Q(\vx)+L\vx=\vu\label{eq:Quad}
\end{align}
where $Q: \mathbb{R}^n \mapsto \mathbb{R}^n$ is a vector-valued quadratic function, that is, there exist matrices $Q_1,\ldots,Q_n $ $\in \mathbb{R}^{n\times n}$, such that
\[[Q(\vx)]_i = \vx^T Q_{i} \vx \quad \forall i \in [n]\]
and $L \in \mathbb{R}^{n\times n}$, $\vu \in \mathbb{R}^n$. 
We are interested in solutions to this system of equations under linear constraints of the form
\begin{align}
(A\vx)_i\leq b_i \quad \forall i \in [n]\label{eq:xLimits}
\end{align}
where we assume that $(A\vx)_i \leq b_i$ for each $i$ is free of redundant constraints and $\vx$ and $\vu$ have the same dimension.
However, the parameter $\vu$ is uncertain and known only up to certain error bounds:
%
\begin{align}
%u^{\min}_i=u_i^*-e_i \leq u_i \leq =u_i^*+e_i=u^{\max}_i \quad \forall i \in [n] \label{eq:uLimits}
u^{\min}_i=u_i^*-e_i \leq u_i \leq u_i^*+e_i=u^{\max}_i \quad \forall i \in [n] \label{eq:uLimits}
\end{align}
%
where $\vu^*$ is a forecast for $\vu$ and $\ve$ denotes the error bounds associated with the forecast. 
For example, in the case of quadratic equations appearing in infrastructure networks like the power grid, $u_i$ represents uncertain power generation or consumption (for example uncertain weather-dependent power sources like solar or wind power). 
In the case of stochastic processes, $\vu^*$ represents an initial state distribution. Further, note that if the polyhedron given by $A\vx \leq b$ is not full dimensional, then there would exist at least a $i\in[1,\dotsb, n]$ such that $x_i$ could be any number as long as satisfies \eqref{eq:Quad}. In this case,for the corresponding uncertain parameter $u_i$, there are no $u_i^{min}$ and $u_i^{\max}$.

\begin{cdef}[Robust Feasibility and Robustness Margin problem]  \label{RobustDef}
  Determine whether for all values of $\vu$ satisfying \eqref{eq:uLimits}, the system of equations \eqref{eq:Quad} has a solution lying within the interior of the set of all  $\vx$ satisfying the constraints in \eqref{eq:xLimits}. 
  If this is true, the system comprised of \eqref{eq:Quad},\eqref{eq:xLimits},\eqref{eq:uLimits} is said to be \emph{robust feasible}. 
  The largest $r$ for which $e_i\geq r \ \forall i~$ with $~e_i>0$ in \eqref{eq:uLimits} and such that the system is robust feasible is defined as the \emph{robustness margin}. 
  See \cref{fig:RobFeas(r)} for a pictorial depiction. 
\end{cdef}

\begin{figure}[htp!]
\begin{center}
  %\includegraphics[scale=1.1, bb=0in 0in 5in 2in]{Figures/RobFeas(revised).png} % {Figures/Rfeas}
  \includegraphics[width=\textwidth]{RobFeas_2.png}
\end{center}
\caption{Illustration of robust feasibility.
The system is robust feasible at the level of $r$ (left), but is not robust feasible at $r' > r$ (right).}
\label{fig:RobFeas(r)}
\end{figure}


\add{There are many practical uncertainty sets developed in previous studies such as ellipsoidal uncertainty \cite{ben1999robust}, cardinally-constrained uncertainty \cite{bertsimas2004robust}, norm uncertainty \cite{bertsimas2004price}, and some other types of uncertainty sets constructed based on probability theory \cite{bandi2012tractable,bandi2014optimal} and data driven approaches \cite{lotfi2022data}.
  We have presented the definition for robustness margin in the most general form so as to capture most scenarios that may be considered by the researcher.
  Note that for a given choice of $r$, our uncertainty set specified as the $\ell_{\infty}$-ball of radius $r$ forms the largest such set---many other commonly used uncertainty sets, e.g., $\ell_1$-ball, $\ell_2$-ball, ellipsoid (with major axis $r$), are subsets of the $\ell_\infty$-ball. }
For instance it very well may be the case that only some of the dimensions of $\vu$ will have margins of uncertainty. Furthermore, one may have need of computing the robustness margin for only a subset of the dimensions of $\vu$ which pertain to problem areas or nodes of particular interest to the research.This manual restriction will of course produce a robustness margin greater than or equal to that obtained by considering all dimensions, which certainly remains an option under the current setting.
