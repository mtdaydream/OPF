% Last updated on March 28, 2022
% Cover Letter for submission to JOptimization (Hindawi)
\documentclass[12pt]{letter}
\newcommand{\mytilde}{\hspace*{-0.02in}\raise.17ex\hbox{$\scriptstyle\mathtt{\sim}$}}

\usepackage{caswsuvletter}
\usepackage{textcomp}

\name{\smallskip Bala Krishnamoorthy}
\email{\href{http://www.wsu.edu/~kbala}{www.wsu.edu/$\mytilde$kbala}}
\telephone{}
%\location{\medskip}
\date{March 28, 2022}

\begin{document}

\begin{letter}{Editorial Board\\
Hindawi Journal on Optimization
%\vspace*{0.1in}
  }

\subject{Submission of our manuscript on Robust Feasibility using Topological Degree Theory.}
%\subject{}

\opening{Dear Colleagues,}

I am submitting our new work on robust feasibility of systems of quadratic equations for possible publication in the Hindawi Journal of Optimization.

The main innovation is the use of results from topological degree theory to define conditions that measure the \emph{margin of robustness} of solutions to such systems.
Using tools from convex analysis and optimization theory, we cast the checking of these conditions as a nonlinear optimization problem.
We then develop \emph{inner} and \emph{outer} bound formulations for this hard optimization problem, which could be solved efficiently in practice to derive lower and upper bounds, respectively, for the robustness margin.
We evaluate our approach on problems from AC power flow equations (OPF).

We believe Optimization is the ideal venue to publish our work.
While robust feasibility and optimization have received much attention from the optimization community, a unified approach that provides both theoretical guarantees for robust feasibility of nonlinear systems and affords efficient computation of the same in practice is still lacking.
Our manuscript lays down the foundations for such a framework, while also demonstrating its applicability in power systems.
The Optimization reader should be directly interested in this work.
Further, our framework is quite general, and could be applied to systems of equations arising in several other application domains that are of interest to the community.

I look forward to hearing about the next steps in the submission and review process.
Please let me know if I can provide any more info or supporting documents.
%\sigspace{6}
%\closing{Sincerely,}
\signclosing{Yours sincerely,}

\end{letter}


\end{document}


