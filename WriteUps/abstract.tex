\begin{abstract}
%  \txgr{Too long right now---could jump directly to our work...}\\
  %  \delete{Systems of quadratic equations appear in several applications from analysis of stochastic processes, machine learning to physical infrastructure systems like the power grid. In recent years, several algorithms have been proposed for the solution of quadratic systems, along with analyses of conditions under which they work. However, in many applicatons, parameters appearing in the quadratic system of equations are not known perfectly. In these cases, it is of interest to study robust solvability of quadratic systems, that is, whether the quadratic system of equations has a solution (within a certain set) for all realizations within a certain error bound of a given nominal value of the parameters.}\\
  We consider the problem of measuring the \emph{margin of robust feasibility} of solutions to a system of nonlinear equations.
%  \txgr{Define robustness margin informally here... }\\
  This problem is turns out to be NP-hard in general.
  We study the special case of a system of quadratic equations, which shows up in many practical applications such as the power grid and other infrastructure networks. 
  We develop approaches based on topological degree theory to estimate bounds on the robustness margin of such systems.
  Our methods use tools from convex analysis and optimization theory to cast the problems of checking the conditions for robust feasibility as a nonlinear optimization problem.
  We then develop \emph{inner bound} and \emph{outer bound} formulations for this optimization problem, which could be solved efficiently to derive lower and upper bounds, respectively, for the margin of robust feasibility.
  We evaluate our approach numerically on standard instances taken from the MATPOWER database of AC power flow equations that describe the steady state of the power grid.
  The results demonstrate that our approach can produce tight lower and upper bounds on the radius of robust feasibility for such instances.
\end{abstract}

